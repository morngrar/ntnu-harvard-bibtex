\documentclass[norsk, a4paper, 11pt]{article}
\usepackage[utf8]{inputenc}
\usepackage[norsk]{babel}       % fornorsker dokumentet
\edef\restoreparindent{\parindent=\the\parindent\relax}
\usepackage{parskip}
\restoreparindent
\usepackage[hyphens, spaces, obeyspaces]{url}
\usepackage{hyperref}
\usepackage{color}
\definecolor{darkblue}{rgb}{0,0,0.5}
\hypersetup{
    colorlinks=true,
    linkcolor=blue,
    filecolor=magenta,      
    urlcolor=blue,
    citecolor=darkblue,
}
\usepackage{csquotes}
\usepackage{graphicx}
\usepackage[T1]{fontenc}
\usepackage{libertine}          % Linux Libertine font
\usepackage{listings}           % syntaxhighlightet kode
\lstset{                        % Setter settings for code listings
    numbers=left,
    numberstyle=\normalsize,
    stepnumber=1,
    numbersep=5pt,
    frame=lines,
    showstringspaces=false,
    aboveskip=6mm,
    belowskip=8mm,
    basicstyle=\small,
}



% Noen custom kommandoer for å ikke bli koko
\renewcommand{\bf}[1]{\textbf{#1}}  % bold font
\renewcommand{\it}[1]{\textit{#1}}  % italics font
\renewcommand{\sc}[1]{\textsc{#1}}  % small caps font
\newcommand{\code}[1]{{\small \texttt{#1}}}  % markup for inline kodeeksempel
\newcommand{\q}{\enquote}                       % forkortelser av qoutekommando
\newcommand{\bq}[3]{\blockquote[{\citealt[#3]{#2}}]{#1}}


% referanser
\usepackage{natbib}
\bibliographystyle{ntnu-harvard}

\title{Eksempel på ntnu-harvard}
\author{Svein-Kåre Bjørnsen}
\date{\today}

\begin{document}
\maketitle

\begin{abstract}
Dette er et eksempel på NTNU harvard stil for kildehenvisning i latex og bibtex. 
Kildefilen inneholder også noen ekstra snutter som gjør dokumentet pent, og latexkoden enklere å skrive.
Gjerne bruk dette som et utgangspunkt for dine egne dokumenter.
\end{abstract}


% kildehenvisning
\section{Kildehenvisning med bibtex}

Tester æøåöä sortering \citep{aa, ae, oe, oo, ee}.

Adams skrev en gang om noe relevant \citep[s.123]{adams1995}. Dette er et eksempel på en parantesreferanse.

\citet{adams1995} skrev en gang noe relevant. Dette er et eksempel på en tekstreferanse.

Et eksempel med mange folk (for å sjekke at referansestilen funker): Vi refererer til mange folk \citep{mangefolk}. Her kommer også et eksempel med færre (1-3) folk: Vi refererer til færre folk \citep{faafolk}

Flere referanser på samme utsagn \citep{faafolk,adams1995}.




\bibliography{example-referanser}
\end{document}
